\section{量子力学的哲学基础}

\subsection{波粒二象性的哲学意义}

光和物质的波粒二象性揭示了现实的多重性质。这种现象表明,我们对事物的认知往往受到观察方式的限制。

\subsection{测量问题与意识}

量子测量问题引发了关于意识在物理现实中作用的讨论。一些解释认为,意识的介入是波函数坍缩的必要条件。

\subsection{量子纠缠与非局域性}

量子纠缠现象展现了超越时空限制的关联性,这与某些玄学传统中的"万物相连"思想不谋而合。 